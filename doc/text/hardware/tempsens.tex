\subsection{Temperatursensor}
Beim Temperatursensor handelt es sich um einen \gls{tmp102} der Firma SparkFun.
Dieser hat einen Einsatzbereich von -40 bis 125°C und wird mit \gls{i2c} angesteuert. \\
Auf dem Sensor stehen folgende Pins zur Verfügung: \\
\gls{scl}, \gls{sda}, \gls{vcc}, \gls{gnd}, \gls{alert}, \gls{add} \\

% TODO: Blockschaltbild vlt?
Der Sensor benötigt \gls{vcc} und \gls{gnd} für die Spannungsversorgung, sowie \gls{scl} und \gls{sda} für die Kommunikation.
Die Pins werden mit dem \gls{rpi} folgendermaßen verbunden: \\
\begin{table}
    \centering
    \begin{tabular}{|l|l|l|l|l|l|}
        \hline
        \textbf{\gls{scl}} & \textbf{\gls{sda}} & \textbf{\gls{vcc}} & \textbf{\gls{gnd}} & \textbf{\gls{alert}} & \textbf{\gls{add}}\\
        \hline
        \hline
        GPIO3 & GPIO2 & & & None & None \\
        \hline
        Pin5 & Pin3 & Pin2 & Pin14 & None & None \\
        \hline
    \end{tabular}
    \caption{\gls{i2c} Pinout Tabelle}\label{i2cpinout}
    \quad
\end{table}


\subsubsection{Hardwareanschluss}
Die Schaltung des digitalen Potentiometers wird direkt an den \gls{rpi} angeschlossen. \\
Die Pins werden folgendermaßen verbunden: \\
\begin{table}
    \centering
    \begin{tabular}{|l|l|l|l|}
        \hline
        \textbf{\gls{mosi}} & \textbf{\gls{miso}} & \textbf{\gls{sclk}} & \textbf{\gls{ss}} \\
        \hline
        \hline
        GPIO10 & GPIO9 & 5V Power & GPIO11 & GPIO8 \\
        \hline
        Pin19 & Pin21 & Pin2 & Pin23 & Pin24 \\
        \hline
    \end{tabular}
    \caption{\gls{spi} Pinout Tabelle}\label{spipinout}
    \quad
\end{table}

\subsubsection{I\textsuperscript{2}C Bus}

\begin{figure}
    \begin{center}
\begin{tikztimingtable}[%
    timing/dslope=0.2,
    timing/.style={x=1.6ex,y=2ex},
    x=1ex,
    timing/rowdist=4ex,
    timing/c/rising arrows,
    timing/name/.style={font=\sffamily\scriptsize},
]
\busref{SCL} & HHL18{C}L;18{C}LHH\\
\busref{SDA} & HL;2D{A6};2D{A5};2D{A4};2D{A3};2D{A2};2D{A1};2D{A0};2D{R/W};3D{ACK};2D{D7};2D{D6};2D{D5};2D{D4};2D{D3};2D{D2};2D{D1};2D{D0};3D{ACK};LLH\\
%
\extracode
\begin{pgfonlayer}{background}
    \begin{scope}[semitransparent ,semithick]
        \vertlines[darkgray,dotted]{0,3.2 ,...,67.0}%
        \draw[draw=black,dashed] (0.8,2) rectangle ++(1.6,-6);%
        \draw node[below] at (1.6, -7ex){\small{Start}};%
        \draw[draw=black,dashed] (64.8,2) rectangle ++(1.6,-6);%
        \draw node[below] at (65.6, -7ex){\small{Stop}};%
    \end{scope}
    \end{pgfonlayer}
\end{tikztimingtable}
\end{center}
\caption[Eine \gls{i2c} Datenübertragung.]{Eine \gls{i2c} Datenübertragung aus 8 Addressbits und 8 Datenbits.
Der Busmaster erstellt das \texttt{SCL} Signal und verschickt die Addresse, das Read/Write Bit und die Datenbits.
Der ausgewählte Slave übernimmt die Acknowledgement bits.
Würde der Slave ein \texttt{NACK} übertragen, müsste der Master die vorheringen 8 Bit wiederhohlen.}
\label{i2c-transaction}
\end{figure}
