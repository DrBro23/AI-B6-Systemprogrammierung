% Robin Prillwitz 2020

\usepackage[acronym, toc]{glossaries}
\usepackage[automake, nonumberlist, nogroupskip]{glossaries-extra}
\makeglossaries
\setabbreviationstyle[acronym]{long-short}

\newacronym{pwm}{PWM}{Pulse width modulation}
\newacronym{i2c}{I\textsuperscript{2}C}{Inter-IC Communication}
\newacronym{spi}{SPI}{Serial peripheral Interface}
\newacronym{ioctl}{IOCTL}{IO-Control}
\newacronym{ipc}{IPC}{Interprocess Communication}

\newacronym{cpu}{CPU}{Central Processing Unit (Prozessor)}
\newacronym{sbc}{SBC}{Single-Board Computer}
\newacronym{cli}{CLI}{Command Line Interface}

\newglossaryentry{rpi}{name={Raspberry Pi}, description={Ein kleiner \gls{sbc} für embedded Anwendungen}}
\newglossaryentry{float}{name={floating point}, description={32 oder 64-bit Gleitkommazahlen}}
\newglossaryentry{int}{name={integer}, description={32 oder 64-bit Ganze, natürliche, Zahlen}}

% cursive acronyms
\renewcommand*{\glstextformat}[1]{\textit{#1}}
% glossary spaceing
\renewcommand\glstreepredesc{\tabto{4cm}}

\makeglossaries
