\section{Motivation}
% Im sechsten Semesters des Studiengangs Angewandte Informatik wird im Fach Systemprogrammierung die Anwendung eines Linux Treibers gelehrt.
% Hierbei entsteht die Möglichkeit die Kenntnisse der Treiberprogrammierung im Rahmen einer Prüfungsstudienarbeit weiter zu vertiefen.

Das Thema dieser Studienarbeit ist die Programmierung eines Gerätetreibers für Linux.
Dieses Projekt implementiert eine Lüftersteuerung mit Temperatursensor.
Zum Einsatz kommen ein \gls{rpi} 4b, ein Noctua \textit{NF-A4x20} 5V \gls{pwm} Lüfter mit vier Pins und ein digitaler Temperatursensor \texttt{TMP102}.
Der Temperatursensor misst die momentane Umbgebungstemperatur.
Je nach Erwärmung wird die Drehgeschwindigkeit des Lüfters beeinflusst.
Hierbei versucht der Lüfter die Temperatur um den Temperatursensor auf normale Umgebungstemperatur herunter zu kühlen.

Das Projekt besteht aus der Applikation im Userspace und dem Treiber im Kernelspace.
Der Treiber kommuniziert mit \gls{i2c} und \gls{spi} mit der Hardwareperipherie.
Die Schnittstelle zwischen Kernelspace und Userspace wird sowohl durch \gls{fops} und \gls{ioctl} ermöglicht.
\autoref{fig:block} beschreibt den abstrakten Aufbau des Projekt.
Im Folgenden Dokument wird der Aufbau von der Software, als auch der Hardware erklärt.

\begin{figure}[h]
    \centering
    \begin{tikzpicture}
        \draw (0,1) node {};
        \draw (0,-10) node {};
        \draw[fill=black!20!white, draw, thick] (-3,0.5) rectangle ++(6cm,-5cm);
        \node[fill=red!40!white, thick, draw, minimum height=1cm, minimum width=6cm] (app) {Applikation (Userspace)};
        \node[minimum height=1cm, minimum width=6cm, below of=app] (fops) {\acrshort{fops} oder \acrshort{ioctl}};
        \node[fill=red!40!white, thick, draw, minimum height=1cm, minimum width=5cm, below of=fops] (driver) {Fanctrl (Kernelspace)};
        \node[minimum height=1cm, minimum width=6cm, below of=driver] (kernel) {Linux Kernel};
        \node[fill=black!20!white, thick, draw, minimum height=1cm, minimum width=6cm, below of=kernel] (hardware) {\gls{rpi} Hardware};
        \node[below= 1cm of hardware] (sensor) {};
        \node[top color=black!20!white, bottom color=red!40!white, thick, draw, minimum height=1cm, minimum width=3cm, left = 0.2cm of sensor] (tmp) {\texttt{TMP201}};
        \node[top color=black!20!white, bottom color=red!40!white, thick, draw, minimum height=1cm, minimum width=3cm, right = 0.2cm of sensor] (mcp) {\texttt{MCP41XXX}};
        \node[fill=red!40!white, thick, draw, minimum height=1cm, minimum width=3cm, below of=mcp] (pwm) {\texttt{555} PWM};
        \node[fill=black!20!white, thick, draw, minimum height=1cm, minimum width=6cm, below = 2cm of sensor] (fan) {Lüfter};
        \draw[black, very thick, ->] (pwm) -- ++(0,-1.15cm) node[above left] {\acrshort{pwm}};
        \draw[black, very thick, <-] (tmp) -- ++(0,-2.15cm) node[left, pos=0.5] {Luft\textregistered};

        \draw[black, very thick, <-] (mcp) -- ++(0,1.15cm) node[below left] {\acrshort{spi}};
        \draw[black, very thick ,<->] (tmp) -- ++(0,1.15cm) node[below left] {\acrshort{i2c}};
        \draw[black, thick, ->] (-4cm, 0) node[left] {Hohe Abstraktion} -- ++(0,-3cm) node[left] {Niedrige Abstraktion};
        \draw[black, thick] (-4cm, -4cm) -- ++(0,-5cm) node[left, pos=0.5] {Hardware};
    \end{tikzpicture}
    \caption[Blockschaltbild des \texttt{Fanctrl} Projekts]{Blockschaltbild des \texttt{Fanctrl} Projekts. Die in rot abgebildeten Lagen sind eigens implementiert.}
    \label{fig:block}
\end{figure}
