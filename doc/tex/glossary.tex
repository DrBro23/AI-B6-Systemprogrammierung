% Robin Prillwitz 2020

\usepackage[acronym, toc]{glossaries}
\usepackage[automake, nonumberlist, nogroupskip]{glossaries-extra}
\makeglossaries
\setabbreviationstyle[acronym]{long-short}

\newacronym{i2c}{I\textsuperscript{2}C}{Inter-IC Communication}
\newacronym{spi}{SPI}{Serial peripheral Interface}
\newacronym{ipc}{IPC}{Interprocess Communication}

\newacronym{cpu}{CPU}{Central Processing Unit (Prozessor)}
\newacronym{sbc}{SBC}{Single-Board Computer}
\newacronym{cli}{CLI}{Command Line Interface}

\newglossaryentry{rpi}{name={Raspberry Pi}, description={Ein kleiner \gls{sbc} für embedded Anwendungen}}
\newglossaryentry{float}{name={floating point}, description={32 oder 64-bit Gleitkommazahlen}}
\newglossaryentry{int}{name={integer}, description={32 oder 64-bit Ganze, natürliche, Zahlen}}

% new 29.05.2022:
\newacronym{scl}{SCL}{Serial Clock bei \acrshort{i2c}}
\newacronym{sda}{SDA}{Serial Data bei \acrshort{i2c}}
\newacronym{vcc}{Vcc}{Voltage at Common Collector}
\newacronym{gnd}{GND}{Ground}
\newacronym{lsb}{LSB}{Least significant bit}
\newacronym{fpu}{FPU}{Floating point unit}
\newacronym{api}{API}{Aplication Programming Interface}
\newacronym{fops}{FOPS}{File Operations}
\newacronym{ioctl}{IOCTL}{I/O-Control}
\newacronym{pwm}{PWM}{Pulse width modulation}

\newglossaryentry{breakout}{name={Breakout Board}, description={Trägerplatine um externe Anschlüsse zu erleichtern}}

\newacronym{mosi}{MOSI}{Master Out, Slave In bei \acrshort{spi}}
\newacronym{miso}{MISO}{Master In, Slave Out bei \acrshort{spi}}
\newacronym{sclk}{SCLK}{Serial Clock bei \acrshort{spi}}
\newacronym{ss}{$\overline{\text{CS}}$n}{Chip Select bei \acrshort{spi}}



% cursive acronyms
\renewcommand*{\glstextformat}[1]{\textit{#1}}
% glossary spaceing
\renewcommand\glstreepredesc{\tabto{4cm}}

\makeglossaries
