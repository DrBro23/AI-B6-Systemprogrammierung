\section{Motivation}
Im sechsten Semesters des Studiengangs Angewandte Informatik wird im Fach Systemprogrammierung die Anwendung eines Linux Treibers gelehrt.
Hierbei entsteht die Möglichkeit die Kenntnisse der Treiberprogrammierung im Rahmen einer Prüfungsstudienarbeit weiter zu vertiefen.

Das Thema der Studienarbeit ist die Programmierung eines Gerätetreibers für eine Lüftersteuerung mit Temperatursensor.
Zum Einsatz kommen ein Raspberry Pi 4b als Basis, ein Noctua NF-A4x20 5V PWM Lüfter mit vier Pins und ein digitaler Temperatursensor TMP102 von SparkFun.

Funktionsweise:
Der Temperatursensor misst seine normale Umgebungstemperatur.
Je nach Erwärmung wird die Drehgeschwindigkeit des Lüfters beeinflusst.
Hierbei versucht der Lüfter die Temperatur um den Temperatursensor auf normale Umgebungstemperatur herunter zu kühlen.
