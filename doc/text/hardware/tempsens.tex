\subsection{Temperatursensor}
\subsubsection{Hardwareanschluss}
\subsubsection{I\textsuperscript{2}C Bus}

\begin{figure}
    \begin{center}
\begin{tikztimingtable}[%
    timing/dslope=0.2,
    timing/.style={x=1.6ex,y=2ex},
    x=1ex,
    timing/rowdist=4ex,
    timing/c/rising arrows,
    timing/name/.style={font=\sffamily\scriptsize},
]
\busref{SCL} & HHL18{C}L;18{C}LHH\\
\busref{SDA} & HL;2D{A6};2D{A5};2D{A4};2D{A3};2D{A2};2D{A1};2D{A0};2D{R/W};3D{ACK};2D{D7};2D{D6};2D{D5};2D{D4};2D{D3};2D{D2};2D{D1};2D{D0};3D{ACK};LLH\\
%
\extracode
\begin{pgfonlayer}{background}
    \begin{scope}[semitransparent ,semithick]
        \vertlines[darkgray,dotted]{0,3.2 ,...,67.0}%
        \draw[draw=black,dashed] (0.8,2) rectangle ++(1.6,-6);%
        \draw node[below] at (1.6, -7ex){\small{Start}};%
        \draw[draw=black,dashed] (64.8,2) rectangle ++(1.6,-6);%
        \draw node[below] at (65.6, -7ex){\small{Stop}};%
    \end{scope}
    \end{pgfonlayer}
\end{tikztimingtable}
\end{center}
\caption[Eine \gls{i2c} Datenübertragung.]{Eine \gls{i2c} Datenübertragung aus 8 Addressbits und 8 Datenbits.
Der Busmaster erstellt das \texttt{SCL} Signal und verschickt die Addresse, das Read/Write Bit und die Datenbits.
Der ausgewählte Slave übernimmt die Acknowledgement bits.
Würde der Slave ein \texttt{NACK} übertragen, müsste der Master die vorheringen 8 Bit wiederhohlen.}
\label{i2c-transaction}
\end{figure}
